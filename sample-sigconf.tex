%
% The first command in your LaTeX source must be the \documentclass command.
\documentclass[sigconf]{acmart}

%
% defining the \BibTeX command - from Oren Patashnik's original BibTeX documentation.
\def\BibTeX{{\rm B\kern-.05em{\sc i\kern-.025em b}\kern-.08emT\kern-.1667em\lower.7ex\hbox{E}\kern-.125emX}}
\long\def\red#1{\textcolor{red}{#1}}

% Rights management information.
% This information is sent to you when you complete the rights form.
% These commands have SAMPLE values in them; it is your responsibility as an author to replace
% the commands and values with those provided to you when you complete the rights form.
%
% These commands are for a PROCEEDINGS abstract or paper.
\copyrightyear{2018}
\acmYear{2018}
\setcopyright{acmlicensed}
\acmConference[Woodstock '18]{Woodstock '18: ACM Symposium on Neural Gaze Detection}{June 03--05, 2018}{Woodstock, NY}
\acmBooktitle{Woodstock '18: ACM Symposium on Neural Gaze Detection, June 03--05, 2018, Woodstock, NY}
\acmPrice{15.00}
\acmDOI{10.1145/1122445.1122456}
\acmISBN{978-1-4503-9999-9/18/06}

%
% These commands are for a JOURNAL article.
%\setcopyright{acmcopyright}
%\acmJournal{TOG}
%\acmYear{2018}\acmVolume{37}\acmNumber{4}\acmArticle{111}\acmMonth{8}
%\acmDOI{10.1145/1122445.1122456}

%
% Submission ID.
% Use this when submitting an article to a sponsored event. You'll receive a unique submission ID from the organizers
% of the event, and this ID should be used as the parameter to this command.
%\acmSubmissionID{123-A56-BU3}

%
% The majority of ACM publications use numbered citations and references. If you are preparing content for an event
% sponsored by ACM SIGGRAPH, you must use the "author year" style of citations and references. Uncommenting
% the next command will enable that style.
%\citestyle{acmauthoryear}

%
% end of the preamble, start of the body of the document source.
\begin{document}

%
% The "title" command has an optional parameter, allowing the author to define a "short title" to be used in page headers.
\title{After Brazil's Data Protection Law: Authorizing the Access to Personal Data in Solid}

%
% The "author" command and its associated commands are used to define the authors and their affiliations.
% Of note is the shared affiliation of the first two authors, and the "authornote" and "authornotemark" commands
% used to denote shared contribution to the research.
\author{Jefferson O. Silva}
\email{silvajo@pucsp.br}
\affiliation{%
  \institution{Pontifical Catholic University of S\~ao Paulo}
  \city{S\~ao Paulo}
  \country{Brazil}
}

\author{Newton Calegari}
\email{newton@nic.br}
\affiliation{%
  \institution{Brazilian Network Information Center - NIC.br}
  \city{S\~ao Paulo}
  \country{Brazil}
}

\author{Diogo Cortiz}
\affiliation{%
  \institution{Inria Paris-Rocquencourt}
  \city{S\~ao Paulo}
  \country{Brazil}
}

%
% By default, the full list of authors will be used in the page headers. Often, this list is too long, and will overlap
% other information printed in the page headers. This command allows the author to define a more concise list
% of authors' names for this purpose.
\renewcommand{\shortauthors}{Silva et al.}

%
% The abstract is a short summary of the work to be presented in the article.
\begin{abstract}
Write the abstract here.

\end{abstract}

%
% The code below is generated by the tool at http://dl.acm.org/ccs.cfm.
% Please copy and paste the code instead of the example below.
%
\begin{CCSXML}
<ccs2012>
 <concept>
  <concept_id>10010520.10010553.10010562</concept_id>
  <concept_desc>Computer systems organization~Embedded systems</concept_desc>
  <concept_significance>500</concept_significance>
 </concept>
 <concept>
  <concept_id>10010520.10010575.10010755</concept_id>
  <concept_desc>Computer systems organization~Redundancy</concept_desc>
  <concept_significance>300</concept_significance>
 </concept>
 <concept>
  <concept_id>10010520.10010553.10010554</concept_id>
  <concept_desc>Computer systems organization~Robotics</concept_desc>
  <concept_significance>100</concept_significance>
 </concept>
 <concept>
  <concept_id>10003033.10003083.10003095</concept_id>
  <concept_desc>Networks~Network reliability</concept_desc>
  <concept_significance>100</concept_significance>
 </concept>
</ccs2012>
\end{CCSXML}

\ccsdesc[500]{Computer systems organization~Embedded systems}
\ccsdesc[300]{Computer systems organization~Redundancy}
\ccsdesc{Computer systems organization~Robotics}
\ccsdesc[100]{Networks~Network reliability}

%
% Keywords. The author(s) should pick words that accurately describe the work being
% presented. Separate the keywords with commas.
\keywords{Solid, Access control, Decentralized web, Frameworks, Guardian}

%
% This command processes the author and affiliation and title information and builds
% the first part of the formatted document.
\maketitle

\section{Introduction}

With the approval of the Brazilian General Data Protection Law (LGPD),\footnote{http://www.planalto.gov.br/ccivil\_03/\_Ato2015-2018/2018/Lei/L13709.htm} several software companies need to redesign the applications they handle the personal data of Brazilian citizens. LGPD is based on the General Data Protection Regulation (GDPR),\footnote{http://data.consilium.europa.eu/doc/document/ST-9565-2015-INIT/en/pdf} which aims at protecting the personal data of EU individuals. In total, around 120 countries adopt comprehensive privacy laws and regulations to protect personal data held by private and public bodies \cite{Banisar2011}. Nevertheless, for the LGPD success, there must be not only fair regulation enforcement but also technological advancements, which potentially includes adopting new software development tools.


Tim Berners-Lee and colleagues propose a platform called Solid (derived from "\textbf{So}cial \textbf{li}nked \textbf{d}ata"), which is can be described as a set of principles, conventions, and tools for building decentralized Web applications [\red{REF}]. An application is considered decentralized when it does not hold users' personal data [\red{REF}]. LGPD considers personal any data that directly or indirectly leads to the identification of a user [\red{REF}]. Solid is based on the principle that users should have full ownership of their personal data. Currently, applications (e.g., Facebook, LinkedIn, Santander) work as "data silos" and all the personal data created in these platforms are controlled by the application companies. In constrast, decentralized Web applications provide complete separation between users' data and the applications that create and consume this data. While users store data in Web-accessible personal online datastores (pods), applications access users data relying as much as possible in W3C standards and Semantic Web technologies [\red{REF}]. Pods are independent of applications, which means that the users can change the application that create or consume their personal data at anytime. Users can also grant or restrict access to their pods using Web Access Controls (WAC).

One implication of using Solid in the context of LGPD is that decentralized Web applications need to respond differently according to the citizenship of the user.



We propose



\section{Background}
In this section, we offer some background on LGPD, decentralized Web, and on access control.

\subsection{Brazilian Data Protection Law}
Altough the Brazilian regulation is strongly inspired by the European GDPR, it has some national specifics, such as cross-border jurisdiction, what implies that the Bill is applicable to any organizations processing personal data of Brazilian residentes, whether it is headquarted in Brazil or not.

LGPD has also included the right of data portability, the right of access to personal data by the owner, and the right of erasure. Differently of the GDPR, which imposes 30 days for the controllers to comply with these requests, the LGPD imposes 15 days.

The Brazilian law also requires companies to nominate a Data Protection Officer (DPO) who will be in charge of monitoring the adoption of best practices for personal data protection and for reporting to the National Data Protection Authority (ANPD).

In a technical perspective, efforts related to the decentralization of the Web help to build systems that are privacy-friendly, respecting user's privacy and in compliance with the regulation.


\textcolor{red}{Ainda nao sei se esse paragrafo fica nessa section ou na proxima.}

\subsection{Access Control}

Access control is typically split into two distinct procedures: authentication, and authorization. While authentication is concerned with determining whether a subject is who it claims to be, authorization is responsible for verifying if the subject is allowed to execute a protected resource. A subject is a term that refers to a user or any other external agent to the system.

\section{Method}


\section{Preliminary Results}


\section{Conclusion}

%
% The acknowledgments section is defined using the "acks" environment (and NOT an unnumbered section). This ensures
% the proper identification of the section in the article metadata, and the consistent spelling of the heading.
\begin{acks}
To Robert, for the bagels and explaining CMYK and color spaces.
\end{acks}

%
% The next two lines define the bibliography style to be used, and the bibliography file.
\bibliographystyle{ACM-Reference-Format}
\bibliography{laweb19}


\end{document}
